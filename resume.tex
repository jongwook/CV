%!TEX TS-program = xelatex
%!TEX encoding = UTF-8 Unicode
% Awesome CV LaTeX Template
%
% This template has been downloaded from:
% https://github.com/posquit0/Awesome-CV
%
% Author:
% Claud D. Park <posquit0.bj@gmail.com>
% http://www.posquit0.com
%
% Template license:
% CC BY-SA 4.0 (https://creativecommons.org/licenses/by-sa/4.0/)
%


%%%%%%%%%%%%%%%%%%%%%%%%%%%%%%%%%%%%%%
%     Configuration
%%%%%%%%%%%%%%%%%%%%%%%%%%%%%%%%%%%%%%
%%% Themes: Awesome-CV

\documentclass[12pt, a4paper]{awesome-cv}

%%% Override a directory location for fonts(default: 'fonts/')
\fontdir[fonts/]

%%% Configure a directory location for sections
\newcommand*{\sectiondir}{resume/}

%%% Override color
% Awesome Colors: awesome-emerald, awesome-skyblue, awesome-red, awesome-pink, awesome-orange
%                 awesome-nephritis, awesome-concrete, awesome-darknight
%% Color for highlight
% Define your custom color if you don't like awesome colors
%\colorlet{awesome}{awesome-pink}
%\definecolor{awesome}{HTML}{CA63A8}
\definecolor{awesome}{HTML}{003399}
%% Colors for text
%\definecolor{darktext}{HTML}{414141}
%\definecolor{text}{HTML}{414141}
%\definecolor{graytext}{HTML}{414141}
%\definecolor{lighttext}{HTML}{414141}

%%% Override a separator for social informations in header(default: ' | ')
%\headersocialsep[\quad\textbar\quad]


%%%%%%%%%%%%%%%%%%%%%%%%%%%%%%%%%%%%%%
%     3rd party packages
%%%%%%%%%%%%%%%%%%%%%%%%%%%%%%%%%%%%%%
%%% Needed to divide into several files
\usepackage{import}


%%%%%%%%%%%%%%%%%%%%%%%%%%%%%%%%%%%%%%
%     Personal Data
%%%%%%%%%%%%%%%%%%%%%%%%%%%%%%%%%%%%%%
%%% Essentials
\name{Jong Wook\hspace{1mm}}{Kim}
\address{50 W 34th st. APT 15A8, New York NY 10001}
\mobile{(+1) 203-747-6860} 
%%% Social
\email{jongwook@nyu.edu}
\homepage{jongwook.kim}
\github{jongwook}
\linkedin{jongwook-kim}
%%% Optionals
\position{Software Engineer{\enskip\cdotp\enskip}Data Scientist{\enskip\cdotp\enskip}Music Informaticist}
\quote{``Vouloir, c'est pouvoir"}


%%%%%%%%%%%%%%%%%%%%%%%%%%%%%%%%%%%%%%
%     Content
%%%%%%%%%%%%%%%%%%%%%%%%%%%%%%%%%%%%%%
%%% Make a footer for CV with three arguments(<left>, <center>, <right>)
\newdateformat{monthyear}{\monthname[\THEMONTH] \THEYEAR}
\makecvfooter
{\monthyear\today}
{Jong Wook Kim~~~·~~~Curriculum Vitae}
{\thepage}

\begin{document}
	%%% Make a header for CV with personal data
	\makecvheader
	
	%%% Import contents
	\cvsection{Employment}
	\begin{cventries}
		\cventry
		{Research Assistant}
		{New York University}
		{New York, NY}
		{April 2016 - Present}
		{
			\begin{cvitems}
				\item Analyzing and improving the collaborative filtering algorithms for music recommendation at iHeartRadio, focusing on the relations between multiple entities using techniques including factorization machines, collective matrix factorization, and hypergraph learning.
			\end{cvitems}
		}
		\cventry
		{Recommender System Engineer \& Data Scientist}
		{Kakao Corporation}
		{Seongnam, S. Korea}
		{June 2014 - August 2015}
		{
			\begin{cvitems}
				\item Developed Apache Spark applications for the company's recommender systems which were gradually being migrated from Apache Hive, and built a framework around Spark called CueSheet, which helps simplify the development cycle of Spark jobs by automating the application packaging and separating the concerns of the implementation and the configuration.
				\item Developed a distributed API server in Scala that delivers personalized recommendations in many domains, which were precomputed and stored in the databases such as HBase and Couchbase. The deployment and scaling was managed using Marathon, and the system supported bucket testing for evaluating different recommendation algorithms on-line.
				\item Designed and developed a distributed and fault-tolerant stream processing framework built atop Apache Kafka, providing stream manipulation primitives such as filter, transformer and joiner for simple and modular implementation of streaming jobs. Joiner, in particular, leveraged in-memory data grids to combine two input streams which may have arbitrarily ordered data  in real-time.
				\item Made a core utility library in Scala for the API servers and Spark applications being developed in the team, focusing on the simple and thread-safe usage. In addition to the basic IO and JSON conversion utilities, the library enabled easier nonblocking access to various endpoints that the team was using, including ElasticSearch, ZooKeeper, OpenTSDB, HBase, Couchbase, and HTTP servers.
				\item Leveraged OpenTSDB and Grafana to record and visualize the real-time statistics of the various kinds of applications running within the distributed architecture. Using the ring buffers, the systems were able to publish real-time metrics without affecting the performance, and they were reflected in the report within seconds, helping quickly identify the trends as well as any unexpected behaviors.
				\item Built and maintained a Jenkins cluster that runs mission-critical Hadoop jobs for the company's recommender systems.
			\end{cvitems}
		}
		\cventry
		{Game Server Platform Developer}
		{NCSOFT Corporation}
		{Seoul \& Seongnam, S. Korea}
		{August 2012 - June 2014}
		{
			\begin{cvitems}
				\item Designed and developed a distributed server platform for Lineage Eternal, the company's upcoming massively multiplayer online role-playing game (MMORPG).
				The platform was composed of a number of fault-tolerant clusters of Java and C++ servers, built using Netty/RxJava and Boost ASIO for the asynchronous and nonblocking IO.
				Its goal was to make the game playable without requiring the users to select and be limited to only one among many game servers.
				\item Built a cross-platform logging library in C++ that resembles SLF4j, to be used in Lineage Eternal's game servers.
				\item Managed the Perforce source control server, which had millions of files being shared among the team of 100+ members.
			\end{cvitems}
		}
		\cventry
		{Game Developer Intern}
		{NCSOFT Corporation}
		{Seoul, S.Korea}
		{May 2011 - Aug 2011}
		{
			\begin{cvitems}
				\item Developed the pathfinding engine of Lineage Eternal, used in both the game client and server, built on top of Havok AI.
			\end{cvitems} 
		}
		\cventry
		{Educational Software Developer}
		{University of Michigan}
		{Ann Arbor, MI}
		{May 2010 - Aug 2010}
		{
			\begin{cvitems}
				\item Developed interactive educational software visualizing molecular movements in Processing, for the thermodynamics class.
			\end{cvitems} 
		}
		\cventry
		{Web Developer \& System Administrator}
		{Techno Press}
		{Daejeon, S. Korea}
		{Jan 2009 - Jun 2009}
		{
			\begin{cvitems}
				\item Developed a website for Techno Press, a publisher of international journals in the field of civil engineering.
			\end{cvitems} 
		}
	\end{cventries}
	
	
	\cvsection{Education}
	\begin{cventries}
		\cventry
		{Ph.D. (in progress) in Music Technology}
		{New York University}
		{New York, NY}
		{Sep 2011 - May 2012, Sep 2016 - Present}
		{
			\begin{cvitems}
				\item Advised by Dr. Juan Pablo Bello, expected graduation in 2019. GPA 3.76/4.00
				\item The area of research includes music information retrieval, automatic music transcription, and music recommendation.
				\item Relevant coursework: Probabilistic Graphical Models, Statistical Natural Language Processing, Convex Optimization, Deep Learning.
			\end{cvitems}
		}
		\cventry
		{M.S. in Computer Science and Engineering}
		{University of Michigan}
		{Ann Arbor, MI}
		{Sep 2009 - Apr 2011}
		{
			\begin{cvitems}
				\item Advised by Dr. Georg Essl, focusing on intelligent systems and interactive music environments. GPA 7.76/9.00
				\item Relevant coursework: Machine Learning, Information Theory, Advanced Compilers, Advanced Computer Network
			\end{cvitems}
		}
		\cventry
		{B.S. in Electrical Engineering, with a Minor in Mathematical Sciences}
		{Korea Advanced Institute of Science and Technology}
		{Daejeon, S. Korea}
		{Sep 2006 - May 2009}
		{
			\begin{cvitems}
				\item {GPA 3.83/4.30, with 3.97/4.30 in electrical engineering and 4.03/4.30 in mathematical sciences.}
			\end{cvitems}
		}
		\cventry
		{Undergraduate Research Opportunity Program in the Deparment of Mechanical Engineering}
		{Imperial College London}
		{London, UK}
		{Jul 2006}
		{
			\begin{cvitems}
				\item {Developed tools for preprocessing data to be used in finite element method software}
			\end{cvitems}
		}
	\end{cventries}

	\cvsection{Honors \& Awards}
	\begin{cvhonors}
		\cvhonor
		{Samsung Scholarship}
		{USD 50,000 per year, for 5 years of the Ph.D. program}
		{}{2010}
		\cvhonor
		{Kwanjeong Scholarship}
		{USD 50,000 per year, for 2 years of the M.S. program}
		{}{2009}
		\cvhonor
		{Meritorious Winner}
		{in the COMAP Mathematical Contest in Modeling 2008}
		{}{2008}
		\cvhonor
		{Korea Presidential Scholarship}
		{KRW 10,000,000 per year, for 4 years of the B.S. program}
		{}{2006}
	\end{cvhonors}

	\cvsection{Publication}
	\begin{cventries}
		\cvpublication
		{Concepts and practical considerations of platform-independent design of mobile music environments}
		{J. W. Kim \& G. Essl, in \textit{Proceedings of the International Computer Music Conference}}
		{July 2011}
	\end{cventries}
	
	\cvsection{Presentations}
	\begin{cventries}
		\cvsimpleentry
		{JuliaCon 2016}
		{MusicProcessing.jl: Music Information Retrieval in Julia}
		{Cambridge, MA}
		{Jun 2016}
		\cvsimpleentry
		{GDG DevFest Korea 2014}
		{Functional Reactive Programing with dart:async}
		{Seoul, S. Korea}
		{May 2014}
		\cvsimpleentry
		{Nexon Developer Conference 2014}
		{Building High Performance Servers with Rx and Functional Reactive Programming}
		{Seongnam, S. Korea}
		{May 2014}
	\end{cventries}
	
	\cvsection{Software Projects}
	\begin{cventries}
		\cvsoftware{https://github.com/apache/incubator-s2graph}{Apache S2Graph}{Contributing to this Apache Incubator project, a distributed OLTP graph database.}
		\cvsoftware{https://github.com/jongwook/MusicProcessing.jl}{MusicProcessing.jl}{A Julia library for music and audio processing.}
		\cvsoftware{https://github.com/jongwook/spark-ranking-metrics}{spark-ranking-metrics}{A Spark implementations of common evaluation metrics for ranking algorithms.}
		\cvsoftware{https://github.com/jongwook/eval-archiver-loader}{eval-archiver-loader}{A simple Scala utility for compiling Scala code on-the-fly into a byte array and loading it later.}
		\cvsoftware{https://github.com/jongwook/lyomiTerm}{lyomiTerm}{MacOS telnet client for legacy Korean bulletin board services (BBS).}		
	\end{cventries}
	
	\cvsection{Skills}
	\begin{cventries}
		\cvskill{Programming}{Scala, Python, Java, C/C++ professionally, and Julia, ObjC, PHP, JS, TS, Matlab for personal/research projects}
		%\cvskill{Music}{Oboe a lot and Piano a little}
		\cvskill{Languages}{Korean (native), English (full professional fluency), Japanese (limited working proficiency)}%
	\end{cventries}%	
\end{document}
